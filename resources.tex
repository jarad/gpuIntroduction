\documentclass{article}

\usepackage{natbib,hyperref}

\title{GPU Resources}

\begin{document} 

\section{Texts}

Two introductory texts to programming in CUDA C are \cite{sanders2010cuda} and\cite{kirk2010programming}. A solid resource for pure C is \cite{kernighan}, and an introductory Python book is \cite{beazley}.

\section{Papers}

A number of statistical algorithms coded on a GPU \citep{lee2010utility}. Bayesian mixture models \citep{suchard2010understanding}.

\section{People}

\subsection{@iastate.edu}

\begin{itemize}
\item Jaroslaw Zola
\item Ryan VanderPlas
\item Wallapak Tavanapong
\end{itemize}

\section{Online}

\subsection{NVIDIA}

\begin{description}
\item[gputools] \url{http://brainarray.mbni.med.umich.edu/Brainarray/Rgpgpu/}
\item[CUDA SDK] \url{http://developer.nvidia.com/gpu-computing-sdk} 
\item[NVIDIA CUDA C Programming Guide] \url{http://developer.download.nvidia.com/compute/DevZone/docs/html/C/doc/CUDA_C_Programming_Guide.pdf}
\item[NVIDIA GPU-Accelerated Libraries: CUBLAS, CULA, CURAND, etc.] \url{http://developer.nvidia.com/cuda/gpu-accelerated-libraries}
\item[NumPy Tutorial] \url{http://www.scipy.org/Tentative\_NumPy\_Tutorial}
\item[SciPy Tutorial] \url{http://docs.scipy.org/doc/scipy/reference/tutorial/general.html}
\item[MatPlotLib] \url{http://matplotlib.org/}
\end{description}



\bibliographystyle{plainnat}
\bibliography{resources}
\end{document}